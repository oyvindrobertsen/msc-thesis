\pagenumbering{roman} 				
\setcounter{page}{1}

\pagestyle{fancy}
\fancyhf{}
\renewcommand{\chaptermark}[1]{\markboth{\chaptername\ \thechapter.\ #1}{}}
\renewcommand{\sectionmark}[1]{\markright{\thesection\ #1}}
\renewcommand{\headrulewidth}{0.1ex}
\renewcommand{\footrulewidth}{0.1ex}
\fancyfoot[LE,RO]{\thepage}
\fancypagestyle{plain}{\fancyhf{}\fancyfoot[LE,RO]{\thepage}\renewcommand{\headrulewidth}{0ex}}

\section*{\Huge Abstract}
\addcontentsline{toc}{chapter}{Abstract}	
$\\[0.5cm]$


As computer systems and networks grow in size and complexity, traditional
top-down engineering techniques are quickly beckoming inadequate for achieving
the desired results. Designing such systems that are robust and resilient, are
able to adapt and self-regulate, can self-reproduce and learn autonomously is a
tremendously hard task. These features are however present in many biological
organisms, wherein they emerge through evolution and development. In the fields
of unconventional and biologically inspired computing, these techniques are used
to create computing systems with the same type of complexity as that found in
biological systems. Cellular Automata (CAs) are an example of a biologically
inspired computing system that can achieve complex, global computation through
local interaction between simple cells at a vast scale.

While the computational capabilities of CAs have been researched extensively,
they have not seen mainstream adaption as a computational paradigm.
Programmability and encoding/decoding of input/output are two major challenges
facing cellular computing systems. Manually specifying the functionality of each
cell in such a way that the desired emergent global behavior of the CA as a
whole is achieved, is infeasible for non-trivial systems. Problem input is
usually encoded in the initial state of the system, and output is decoded from
the state after the system has been simulated some amount of time.

The Cellular Automata Research Platform (CARP) is an FPGA-based implementation
of a developmental cellular architecture, aiming to facilitate research into use
of artificial development and evolution to create cellular computing systems. At
an abstract level, it implements a dynamical system with dynamical structure
(DS)$^2$, a system for which both behavior and structure are emerging
properties. Behavior influences further structural development and vice versa.

In this thesis, the platform is extended to incorporate the developmental CA
into a reservoir computing architecture. Reservoir computing (RC) is a novel
approach to machine learning in which temporal input is imposed as perturbations
on a dynamic reservoir and output is read out by performing a linear
classification of the reservoir state some time after the initial perturbation.
By combining RC and developmental CAs, the CARP system solves many of the issues
relating to programming of and I/O encoding/decoding with cellular computing
systems. It also opens up new possibilities for the developed organisms to adapt
and learn based on their environment. The CARP platform has been extended with a
reconfigurable readout layer implemented as a spiking neural network (SNN) that
classifies the dynamics of the reservoir, the developmental CA. An SNN is chosen
to allow the system operate entirely in the spiking domain, as input data and
the dynamic behavior of the reservoir is already spiking in nature.

The extended platform has been verified through extensive testing, both in
simulation and end-to-end on actual hardware.


\cleardoublepage
\pagestyle{fancy}
\fancyhf{}
\renewcommand{\chaptermark}[1]{\markboth{\chaptername\ \thechapter.\ #1}{}}
\renewcommand{\sectionmark}[1]{\markright{\thesection\ #1}}
\renewcommand{\headrulewidth}{0.1ex}
\renewcommand{\footrulewidth}{0.1ex}
\fancyfoot[LE,RO]{\thepage}
\fancypagestyle{plain}{\fancyhf{}\fancyfoot[LE,RO]{\thepage}\renewcommand{\headrulewidth}{0ex}}

\section*{\Huge Sammendrag}
\addcontentsline{toc}{chapter}{Sammendrag}	
$\\[0.5cm]$

Etter hvert som datasystemer og nettverk vokser i størrelse og kompleksitet, er
tradisjonelle topp-ned teknikker raskt blitt utilstrekkelige for å oppnå de
ønskede resultatene. Å designe robuste systemer som er i stand til
å tilpasse seg og selvregulere, som kan selvreprodusere og lære autonomt er en
utrolig vanskelig oppgave. Disse egenskapene er imidlertid tilstede i mange
biologiske organismer, hvor de fremkommer gjennom evolusjon og utvikling. I
forskningsområdene ukonvensjonell og biologisk inspirert databehandling brukes disse
teknikkene til å lage beregningsarkitekturer med samme type kompleksitet som
det som finnes i biologiske systemer. Cellulære Automata (CA) er et eksempel på
et biologisk inspirert datasystem som kan oppnå komplisert, global beregning
gjennom lokalt samspill mellom enkle celler i stor skala.

Mens beregningsevnen til CAer har blitt forsket på over lengre tid, har de ikke
blitt tatt i bruk som et beregningsmessig paradigme i stor skala.
Programmerbarhet og koding/dekoding av data inn og ut er to store utfordringer
for cellulære datasystemer. Manuell spesifisering av funksjonaliteten til hver
celle på en slik måte at den ønskede fremtredende globale oppførselen til CAen
som helhet oppnås, er praktisk ugjennomførbart for ikke-trivielle systemer.
Input data til problemet er vanligvis kodet i systemets innledende tilstand, og
data ut dekodes fra tilstanden etter at systemet har blitt simulert over tid.

Cellular Automata Research Platform (CARP) er en FPGA-basert implementasjon av
en cellulær arkitektur, med sikte på å fasilitere forskning på bruk av kunstig
utvikling og evolusjon for å skape cellulære datasystemer. På et abstrakt nivå
implementerer plattforment et dynamisk system med dynamisk struktur (DS)$^2$, et
system hvor både oppførsel og struktur er fremvoksende egenskaper. Atferd
påvirker videre strukturell utvikling og omvendt.

I denne oppgaven blir plattformen utvidet til å inkorporere kunstig utviklede
CAer i en reservoir computing arkitektur. Reservoir Computing (RC) er en ny tilnærming til
maskinlæring hvor temporal inndata påføres som forstyrrelser på et dynamisk
reservoar, og data leses ut ved å utføre en lineær klassifisering av
reservoar-tilstanden en stund etter den første forstyrrelsen. Ved å kombinere RC
og utviklings-CA, løser CARP-systemet mange av problemene knyttet til
programmering av og I/O enkoding/dekoding av cellulare beregningsarkitekturer. Det
åpner også nye muligheter for de utviklede strukturene til å tilpasse seg og
lære basert på deres miljø. CARP-plattformen er utvidet med et rekonfigurerbart
avlesingslag implementert som et spiking neural network (SNN) som
klassifiserer dynamikken i reservoaret, en cellulær struktur under utvikling. En SNN er valgt for å
tillate at systemet opererer helt i spiking-domenet, ettersom data inn og den
dynamiske oppførelsen av reservoaret allerede er spikes.

Den utvidede plattformen har blitt verifisert gjennom omfattende testing, både i
simulering og ende-til-ende på faktisk maskinvare.

\cleardoublepage