%===================================== CHAP 1 =================================

\chapter{Introduction}

In recent years, research into computation using non-traditional physical
mediums and paradigms, so called unconventional computing, has seen increased
interest. With challenges currently facing traditional architectures, such as
the von Neumann bottleneck and ensuring continued scalability and reliability,
unconventional computation presents possible solutions from a new perspective.
\todo{reword these sentences? same as spec proj} Biologically inspired
computing is one approach to unconventional computation. It seeks to apply
evolution, developement and other biological processes onto the design of both
computer architectures and artificial intelligence systems. A common trait for
many biological systems is that complex behaviour emerges from local
interactions between simple units, whereas traditional computer architectures
have been designed in a top-down fashion, composing complex modules and
directing how information should flow between them. Alongside complex emergent
behavior, biological systems often exhibit abilities such as self-reproduction,
self-regulation and strong adaptability. These are desireable abilities in
computing systems, but they are hard to implement.

Cellular computing, introduced by Sipper in~\cite{Sipper1999}, is one example of
a paradigm utilizing a bottom-up design methodology. Consisting of three core
principles; simplicity, vast parallelism and locality, cellular computing seeks
to harness the emergence of complex global behaviour from local interaction
between simple cells at a large scale. One of the central problems with the
paradigm is how one should go about designing/programming it. Specifying the
functionality and potentially the connectivity of each cell manually while
ensuring that the desired behaviour is achieved at a global level, is
infeasible, so a different approach is needed. A potential solution is to
automate the design through artificial evolution~\cite{Sipper2004}.

The POE-model, introduced by Sanchez et al~\cite{Sanchez1997}, is a taxonomy
commonly used to describe bio-inspired design methodologies using three categories:
phylogeny, ontogeny and epigenesis. Phylogenesis relates to evolution, ontogeny
encompasses systems that mimic biological development and epigenetic systems
adapt to environmental change. These categories are not mutually exclusive. The
use of artificial evolution to program a cellular computing system is an example
of an phylogenetic system.

In artificial evolution, an individual represents a potential solution to some
problem. In the case of using evolution to design cellular computing systems,
the individual, or genome, has to encode both the behavior of each cell as well
as the system, or organism, as a whole in order to represent a complete
solution. This means that the size of the genome grows linearly with the size of
the organism. In natural evolution, the genome has a different role. It serves
as a set of rules governing the growth and development of cells at a local
level, based on the types of surrounding cells and environmental feedback. This
can be incorporated into the cellular computing paradigm as an ontogenetic
aspect~\cite{Tufte2005a}. By evolving the developmental rules instead of the
system as a whole, arbitrarily large and complex organisms can develop from a
single cell based on a genome of fixed size. Systems that separate growth and
behavior in this manner are called dynamical systems with dynamical structure
($DS^2$)~\cite{Tufte2016}.

The Cellular Automata Research Platform (CARP) is a long-running project at
NTNU, dedicated to developing hardware that facilitates research into artificial
evolution and development of cellular computing architectures. Based on the
Virtual SBlock architecture presented by Haddow and Tufte~\cite{Haddow2000a},
the system consists of programmable cells laid out in a regular one-, two- or
three-dimensional grid, where each cell is connected to the cells in the von
Neumann neighborhood around it. Cells can be in one of two states, either alive
or dead. Based on their type, cells are programmed with a look-up-table (LUT)
governing the transition between states based on the states of neighboring cells
and the state of the cell itself. All cells are updated synchronously in
discrete time steps. Development is simulated as a separate process, wherein
cells transition between types using a LUT of development rules, taking both
states and types of neighboring cells as input. This process also happens
synchronously and in descrete development steps.

The CARP system is implemented on reconfigurable hardware, an FPGA, and is
controlled by a program running on a host computer. Typically, the host program
will implement the phylogenetic aspect of the system by evolving a population
wherein each individual is a set of developmental rules. The CARP hardware is
used to assess the fitness of each individual through development and simulation
of the cellular organism. 

Reservoir Computing (RC) is an exiting, new field of research within machine
learning and intelligent systems. RC-systems work by imposing input data as
perturbations on a dynamic system (the reservoir), and performing a linear
classification of the reservoir state some time after the initial perturbation.
Feedback from the classifier is routed back into the reservoir to allow it to
regulate and adapt based on its own performance. In the specialization project
leading up to this thesis, a proof of concept of a cellular reservoir and a
readout layer implemented as a spiking neural network was simulated in software,
with positive results. A common problem with cellular computing systems is that
due to their very nature, it is often hard to formulate problems correctly and
to interpret the dynamics of the system as answers to those problems. Combining
the developmental, cellular architectures of the CARP system with the abstract
computational concept of the RC paradigm will yield a more consistent framework
for applying cellular computing to real-world problems. 

In this thesis, the CARP system has been extended to include a trainable
readout-module based on spiking neural networks which processes the dynamic
behavior of the cells in real-time. It has also been ported to run on new
hardware and the codebase has been partially ported to Chisel, a hardware
definition domain specific language implemented in Scala.

\section{Outline}

This thesis is organized in the following chapters:

\begin{itemize}
\item \textbf{Chapter 2 - Background}: An overview of theoretical concepts on which the
  work presented in this thesis is built upon. Also gives an introduction to
  FPGA technology as well as an overview of some related work.
\item \textbf{Chapter 3 - Previous Work}: A review of the history of the CARP project.
\item \textbf{Chapter 4 - Platform}: Information regarding the physical hardware used to
  run the platform and the toolchains used to develop the project.
\item \textbf{Chapter 5 - Implementation}: An overview of the implemented system and its
  constituent parts.
\item \textbf{Chapter 6 - Verification}: Descriptions of tests used to verify system functionality.
\item \textbf{Chapter 7 - Discussion}: A review of challenges with the system implemented
  in this thesis and possible future work.
\item \textbf{Chapter 8 - Conclusion}: Conluding remarks.
\item \textbf{Appendices}
\end{itemize}

\cleardoublepage
%%% Local Variables:
%%% mode: latex
%%% TeX-master: "../thesis"
%%% End:
