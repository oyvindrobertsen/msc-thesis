\chapter{Discussion}

The proof of concept experiment in Section~\ref{sec:selfreg-example} gives a
small glimpse into the possibilities opened up by adding the Readout module
to the CARP system. While the example only showcases how processing the dynamics
of the CA allows for adaption in the developmental process, more complex
configurations would, hopefully, allow for cellular organisms that not only are
capable of advanced computations, but that are able to adapt and learn from
environmental feedback in ways not previously possible.

\section{Resource Usage}

As shown in Table~\ref{tbl:resource-usage}, the Virtex-7 FPGA on the Convey
coprocessor gives the CARP system quite a bit of room to grow. A design with an
SBlock Matrix with dimensions $96\times96 = 9216 cells$ uses $\sim 20\%$ of LUTs
and $\sim 24\%$ of BRAMs available. In terms of flat quantities, the number
of LUTs used has increased significantly in the new design. By comparing designs
with similar SBlock Matrices, it becomes apparent that this is due to the added
overhead of the Convey PDK Application Engine design into which the CARP design
is placed.


\renewcommand{\arraystretch}{1.2}
\begin{table}[ht]
  \centering
  \begin{tabularx}{\linewidth}{X|X|X|X|X|X|X}
    Cell Count & LUTs Total & LUTs \% & Registers Total & Registers \% & BRAMs Total & BRAMs \%\\
    \hline
    $8\times8$ & 194624 & 15.93 & 208690 & 8.54 & 227.5 & 18.91\\ 
    \hline
    $64\times64$ & 224442 & 18.37 & 221061 & 9.05 & 280.0 & 23.28\\
    \hline
    $72\times72$ & 230094 & 18.84 & 223355 & 9.14 & 286.5 & 23.82\\
    \hline
    $96\times96$ & 251219 & 20.56 & 231616 & 9.48 & 308.5 & 25.64\\
    \hline
    $4\times4\times4$ & 195360 & 15.99 & 209171 & 8.56 & 232.5 & 19.33\\
    \hline
    $16\times16\times16$ & 249320 & 20.41 & 218109 & 8.93 & 261 & 21.70\\
  \end{tabularx}
  \caption{FPGA Resource usage of the CARP system}\label{tbl:resource-usage}
\end{table}

\renewcommand{\arraystretch}{1}


\section{Challenges}

During testing, it was discovered that the Xilinx FPGA flow crashes during
synthesis of large SBlock Matrices. A 1x96x96 large design completes without
problems, a 1x128x128 design however stalls while synthesising SBlocks and
receives a segmentation fault after a few hours. Due to time constraints and the
fact that this issue did not hinder the implementation of new functionality, it
has not been investigated thoroughly. To fully utilize the resources available on the
Convey coprocessor it will have to be resolved however.

The modules that remain implemented in VHDL use custom types extensively. This
makes it difficult to simulate the CARP hardware system as a whole, as no
available simulation tools support mixed-language Verilog/VHDL simulation of
designs using non-standard types. Being unable to simulate the system end-to-end
means that a full synthesis of the system is required in order to test even the
smallest of changes. A full synthesis takes upwards of two hours, even for small
matrix dimensions, resulting in a very long feedback cycle. The optimal way to
resolve this, would be to port the remaining modules to Chisel, and implement
end-to-end test utilizing the C++ simulator generated by the framework. A second
strategy would be to implement custom simulation modules for all modules
blackboxed by Chisel. This would also allow for end-to-end tests in the Chisel
simulator, but in terms of workload required be comparable to actually porting
the system completely. A third solution would be to remove all custom types from
the VHDL parts of the design. This would certainly be quicker than a full port,
but having a fragmented codebase is definitely less desirable in the long term.

\section{Future Work}

To further extend the platform along the epigenetic axis, it is necessary to
allow for other inputs and perturbartions to the CA than the feedback from the
Readout module. A possible solution would be to implement a generic type of
InputCell that, similar to the FeedbackCell used in this thesis, has state
determined entirely by an outside source. That outside source could potentially
be a bitstream stored on the on-board memory on the Convey coprocessor.

In the example presented in Section~\ref{sec:selfreg-example}, configuring the
SNN in the Readout module was trivial to do by hand in order to illustrate the
potential capabilities. In order to effectively utilize the Readout module in
more advanced experiments, an efficient training algorithm will have to be
implemented. In general, training spiking neural networks is a hard task. In the
specialization project report in Appendix~\ref{app:spec-proj-report}, an
evolutionary algorithm is used to guide a search through the space of possible
network configurations. This technique is both slow and it does not converge
reliably. A more efficient solution would be to implement a version of the
SpikeProp algorithm \cite{Bohte2002} adapted to the simplified spiking neural
networks modeled in the Readout module. Another possible approach is fully
on-line training based on spike-timing-dependent plasticity \cite{Markram2012}.
Here, weight tuning happens locally inside each neuron, based on the timing of
incoming spikes relative to when the neuron fires. This approach is used in IBMs
Truenorth computer \cite{6055293}, discussed in Section~\ref{sec:truenorth}.

\cleardoublepage
%%% Local Variables:
%%% mode: latex
%%% TeX-master: "../thesis"
%%% End:
