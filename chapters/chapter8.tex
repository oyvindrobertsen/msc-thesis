\chapter{Conclusion}

In this thesis, the Cellular Automata Research Platform (CARP) has been
augmented with new functionality. A flexible, reconfigurable Readout module has
been added to the system, designed to perform real-time classification of
temporal behavior in a dynamical system with dynamical structure (DS)$^2$. The
Readout module serves as an effecient way to allow the behavior of cellular
computing systems to be interpreted as output. The output from the Readout
module is also fed back into the dynamical system being simulated, as a form of
environmental feedback that both the behavior and structure of the system can
utilize to self-regulate.

With this new functionality, the CARP systems is opened up to being used for
research that encompasses all three axis of the POE-ontology. Systems develop
and adapt over time (Ontogeny) based on evolved developmental rules
(Phylogenesis) and feedback from the environment (Epigenesis).

The platform has also been ported to run on new state-of-the-art hardware. It
has also been partially ported to a more modern implementation language, Chisel.
The new platform has been extensively tested through both unit tests of
simulated modules and functional tests of the entire system end-to-end. The new
hardware should allow the current implementation to scale up to $\sim
192\times192$ CA cells in 2D configurations and $\sim 24\times24\times24$ in 3D.

All in all, this thesis provides a flexible and powerful framework for research
into cellular computing, artificial development and evolution, and regulation
and design of (DS)$^2$ systems.

\cleardoublepage