\chapter{Interface Specifications}
\label{app:interfaces}

\section{Advanced Peripheral Bus (APB)}

APB is a bus specification by ARM designed for low bandwidth communication between register interfaces.
Table~\ref{tbl:apb-signals} lists its constituent buses and signals.

\begin{table}
    \begin{tabular}{p{0.2\linewidth}|p{0.2\linewidth}|p{0.5\linewidth}}
    \hline
    Signal   & Source                 & Description                                                                                                                                                                                                                                                                       \\ \hline
    \textbf{PCLK}     & Clock Source           & Clock. The rising edge of PCLK times all transfers on the APB.                                                                                                                                                                                                                    \\ \hline
    \textbf{PRESETn}  & System bus equivalent  & Reset. The APB reset signal is active LOW. This signal is normally connected directly to the system bus reset signal.                                                                                                                                                            \\ \hline
    \textbf{PADDR}    & APB bridge             & Address. This is the APB address bus. It can be up to 32 bits wide and is driven by the peripheral bus bridge unit.                                                                                                                                                               \\ \hline
    \textbf{PPROT}    & APB bridge             & Protection type. This signal indicates the normal, privileged, or secure protection level of the transaction and whether the transaction is a data access or an instruction access.                                                                                               \\ \hline
    \textbf{PSELx}    & APB bridge             & Select. The APB bridge unit generates this signal to each peripheral bus slave. It indicates that the slave device is selected and that a data transfer is required. There is a PSELx signal for each slave.                                                                      \\ \hline
    \textbf{PENABLE}  & APB bridge             & Enable. This signal indicates the second and subsequent cycles of an APB transfer.                                                                                                                                                                                                \\ \hline
    \textbf{PWRITE}   & APB bridge             & Direction. This signal indicates an APB write access when HIGH and an APB read access when LOW.                                                                                                                                                                                   \\ \hline
    \textbf{PWDATA}   & APB bridge             & Write data. This bus is driven by the peripheral bus bridge unit during write cycles when PWRITE is HIGH. This bus can be up to 32 bits wide.                                                                                                                                     \\ \hline
    \textbf{PSTRB}    & APB bridge             & Write strobes. This signal indicates which byte lanes to update during a write transfer. There is one write strobe for each eight bits of the write data bus. Therefore, PSTRB[n] corresponds to PWDATA[(8n + 7):(8n)]. Write strobes must not be active during a read transfer.  \\ \hline
    \textbf{PREADY}   & Slave interface        & Ready. The slave uses this signal to extend an APB transfer.                                                                                                                                                                                                                      \\ \hline
    \textbf{PRDATA}   & Slave interface        & Read Data. The selected slave drives this bus during read cycles when PWRITE is LOW. This bus can be up to 32-bits wide.                                                                                                                                                         \\ \hline
    \textbf{PSLVERR}  & Slave interface        & This signal indicates a transfer failure. APB peripherals are not required to support the PSLVERR pin. This is true for both existing and new APB peripheral designs. Where a peripheral does not include this pin then the appropriate input to the APB bridge is tied LOW.      \\ \hline
    \end{tabular}
    \caption{Signals and buses in the Advanced Peripheral Bus protocol.}
    \label{tbl:apb-signals}
\end{table}

\clearpage


\section{MemPort Bus}

The MemPort Bus is a generic memory interface with a request/response style
protocol. It consists of two MemBuses wrapped in ready-valid interfaces, one for
requests and one for responses. A MemBus has data, address and write enable fields.
A full list of signals is given in table~\ref{tbl:memport-signals}.

\begin{table}[H]
    \begin{tabular}{p{0.2\linewidth}|p{0.2\linewidth}|p{0.5\linewidth}}
    \hline
    Signal          & Source    & Description                                                                        \\ \hline
    \textbf{req\_addr}       & Requester & Request address.                                                                   \\ \hline
    \textbf{req\_data}       & Requester & Request data. Only used for write requests.                                        \\ \hline
    \textbf{req\_write}      & Requester & Request type. High for write, low for read.                                        \\ \hline
    \textbf{req\_ready}      & Responder & Responder ready. Asserted by responder when it is ready to receive a new request.  \\ \hline
    \textbf{req\_valid}      & Requester & Request valid. Asserted by requester when a new request is dispatched.             \\ \hline
    \textbf{resp\_addr}      & Responder & Response Address.                                                                  \\ \hline
    \textbf{resp\_data}      & Responder & Response data.                                                                     \\ \hline
    \textbf{resp\_write}     & Responder & Response type.                                                                     \\ \hline
    \textbf{resp\_ready}     & Requester & Requester ready. Asserted by requester when it is ready to receive a new response. \\ \hline
    \textbf{resp\_valid}     & Responder & Responder valid. Asserted by responder when a new response is dispatched.          \\ \hline
    \textbf{flush}           & Requester & Flush memory.                                                                      \\ \hline
    \textbf{flush\_complete} & Responder & Flush complete.                                                                    \\ \hline
    \end{tabular}
    \caption{Signals and buses in the MemPort protocol.}
    \label{tbl:memport-signals}
\end{table}